\section{Introduction}


When researching in security analytics and network security areas, it is very important to have appropriate testbed. The appropriate testbed includes appropriate data in needed amount and the appropriate test network environment. But often researchers do not have one of them or both of them.  

In case with data, the lack of data is a result of the fact that mostly the huge amount of data belongs to industrial companies, but they do not hurry to share them because of security and privacy reasons. The limitation with data makes data for security research is more inaccessible. And even if there is possibility to get such information, this information should be anonymized before exporting for analysis.

In cases with the test network environment, researches need to create and configure the test network environment manually, because the test network for IT security research is specific. The specifics of the test network environment for IT security research means that often needed to have a potentially vulnerable environment and that means that old and vulnerable software applications must be installed. This restriction complicates the ability to use many existing IT automation systems such as Chef\footnote{Chef. http://www.getchef.com} and Puppet\footnote{Puppet. http://puppetlabs.com}. These problems are common to many researchers, but for researchers in the security analytics and network security areas, these issues become challenges. 

Within the IT security group of the chair "Internet Technologies and Systems" we also face with problems of the lack of testbeds. We are trying to solve them by automation the process of testbed creation. In the second section of the report we show how we solve the problem of the lack of data by the simulation of user behavior and in the third section of the report we show how we plan to solve the problem of the absent of testbed.



%The scientific world has some problems that slow down research progress. Some of them are obvious. One of them is the lack of data. Researchers need data for research. The huge amount of data belongs to industrial companies, but they do not hurry to share them because of security and privacy reasons. Researchers have to generate and simulate data. Further, many researchers need to create and configure a test network environment for research. It means that they have to run virtual machines, install software applications and configure the network. The third problem is that researchers need to simulate some activities on the test environment such as user behavior, opening connections, running services and so on. These problems are common to many researchers, but for researchers in the security analytic and network security areas, these issues become challenges. The data for security research is more inaccessible than others. For example, domain controller logs often contain personal information, e.g. user Id, time of login and logout events. This information is often an object of data privacy and should be anonymized before exporting for analysis. And it actually means that almost there is not a chance to get this data. The test environments for research in the security area are usually more specific. Every often it is needed to have a potentially vulnerable environment, and that means that old and vulnerable software applications must be installed. This restriction complicates the ability to use many existing IT automation systems such as Chef\footnote{Chef. http://www.getchef.com} and Puppet\footnote{Puppet. http://puppetlabs.com}. The remainder of the report covers ideas, ways of overcoming some mentioned issues and an overview of research projects.

%The remainder of the paper is structured as follow: 


%, because often researchers have to create testbeds manually. In the IT security group we also face with these problems