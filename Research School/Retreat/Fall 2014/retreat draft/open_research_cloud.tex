%\documentclass{article}
%\begin{document}


\section{Open Research Cloud}

\subsection{Related Work}
The Open Science Data Cloud (OSDC) is a petabyte scale science cloud for researchers to manage, analyze and share their data and to get easy access to data from other scientists. Currently OSDC hosts about 700 TB of data. Also OSDC has some other projects for collaboration work on cloud. [!? link https://www.ci.uchicago.edu/research-centers/open-science-data-cloud]

The Massachusetts Open Cloud (MOC) is a new public cloud, designed and implemented in Massachusetts as the first "Open Cloud eXchange" (OCX). The Open Cloud Exchange (OCX) is envisioned as a public cloud marketplace in which many stakeholders, rather than just a single provider, participate in 
implementing and operating the cloud. [!? link http://www.cs.bu.edu/fac/best/res/papers/ic14.pdf]


NeCTAR (National eResearch Collaboration Tools and Resources) is a computing resource for all Australian researchers where they can develop and deploy applications and collaborate in a uniform environment with controlled sharing of data. [!? link http://nectar.org.au/]



\subsection{OpenStack as a basement of Open Research Cloud}
OpenStack [!? link] is open source software for building private and public clouds. OpenStack is not the first open source software for building clouds. There are also OpenNebula [!? link], Eucalyptus [!? link], Cocaine [!? link], CloudStack [!? link] and others. There are a lot of benchmarks of comparison different cloud software [!? link "http://en.wikipedia.org/wiki/" ] . But why the OpenStack must be chosen to build Open Research Cloud. There are some reasons. First, there are success cases of using OpenStack as a basement for building research clouds such as MOC and NeCTAR. The second, a lot of big industrial companies stay behind OpenStack such as NASA, Rackspace, Ubuntu, HP, IBM and others [ !? link http://www.openstack.org/foundation/companies/]. The third is the huge community. 19 000 people in 143 countries can be wrong. If people and big companies trust OpenStack it means there are reason. The reasons are open source, flexible architecture and service based architecture, plugable service and even more.  

What is Open Research Cloud? It could mean different things just look at Related Works mentioned above. But in any case it makes researches easier. In our case when we speak about Open Research Cloud we mean that it is some Research Infrastructure as a Service with some additional research tools The tools could be used for sharing scientific data, collaborating with other research projects, connection to the other tool by specified API. 



\subsection{OpenStack at HPI}
To learn OpenStack and learn how OpenStack could be used as a basement for Open Research Cloud the infrastructure was deployed at HPI servers. The infrastructure includes services: Identity (Keystone), Compute (Nova), Image service (Glance), Networking (Neutron), Block Storage (Cinder), Object Storage (Swift), Dashboard (Horizon), Orchestration (Heat), Telemetry (Ceilometer). The deployment was the first attempt to learn OpenStack. In the brackets are code names of services. The next step will be attempt to make a research by using OpenStack environment. 



\subsection{Security Lab Generator based on OpenStack}
OpenStack is open source Infrastructure as a Service that provides capability to run any complex network environments. The idea of Security Lab based on OpenStack (CloudSLG) is migrate current SLG into the cloud. It will allow to use wide functionality of IaaS in SLG interests. It means that will not need care about how to run, configure the network, how to connect to running instance. It will allow to concentrate to solving more research questions such as analyzing user activities, finding user attack graph, reporting analytic statistic. Also, migration SLG into the cloud could be the first step of making Open Research Cloud.    





%\end{document} 