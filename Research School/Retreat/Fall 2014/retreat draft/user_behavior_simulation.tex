%\documentclass[english, inproceedings]{polze}
%\usepackage{paralist} % Used for the compactitem environment which makes bullet points with less
%\usepackage{includegraphics} % Images
%\begin{document}

%%VARIABLES

\newcommand{\myAuthor}{Aragats Amirkhanyan}
\newcommand{\myGroup}{Internet Technologies and Systems}
\newcommand{\myOrganisation}{Hasso-Plattner-Institut}
\newcommand{\myEmail}{aragats.amirkhanyan@hpi.de}
\newcommand{\myEvent}{Fall 2014 Workshop}
\newcommand{\myTitle}{Automation of the Testbed Creation for Research in Security Analytics and Network Security Areas}
\newcommand{\myShortTitle}{Automation of the Testbed Creation for Research in Security Area}

\newcommand{\myProjectName}{Security Lab Generator}
\newcommand{\myNDS}{Network Designer Service}
\newcommand{\myPLT}{Provision Language Translator}
\newcommand{\myPS}{Provision System}
\newcommand{\myNSA}{Network Security Analyzer}
%\author{\myAuthor}
\group{\myGroup}
\organisation{\myOrganisation}
\email{\myEmail}
\event{\myEvent}
\title{\myTitle}
\maketitle
\title{\myShortTitle}





\section{Simulation of users behaviors}
There is task to analyses the behavior of users to predict attacks or abnormal behavior. There are some algorithms for Attack Prediction based on Machine Learning [http://www.csjournals.com/IJCSC/PDF1-2/51..pdf]. The algorithms could be used for research purposes, in this task the big problem is lack of data. The lack of data force to write tools for generating data by simulation of user behaviors. 

The Scenario in our case is the description of the network infrastructure, information about users, and scenario of their behavior. To analysis user behaviors by predict algorithms we must have normal scenario and abnormal scenario.  
  
\subsection{Environment description}

Description of a network:
\begin{compactitem}
\item 4 computers.
\item Domain controller. 
\item Wiki server.
\item Database (DB) server.
\end{compactitem}
Users: 
\begin{compactitem}
\item Ivanov
\item Petrov
\item Smirnov
\item Admin
\end{compactitem}
     
\subsection{Normal scenario}
All users has an access to their computer, Wiki server and Database server. Admin has an access to any computer such as user's computers, Wiki server, DB server, Domain Controller and his own computer.  In case of normal scenario ordinal users usually login to their computers by their credentials. They do it several times per day. Petrov and Smirnov during the workday go to Wiki server. Ivanov does not go there despite that he has an access. As was mentioned above all users  have access to DB Serve, but nobody uses it. Admin usually login to his computer and during the workday he connects to Domain Controller, DB server and Wiki server several time. Is is clear that other user do not connect to Domain Controller because they do not have an access. The visualized description of normal behavior you can see on the Figure 1: Normal scenario. All black lines show  accesses to computers for particular users. The black solid lines illustrate accesses which are used by users. The black dash lines show accesses which are not used by users in the normal behavior despite that they have an access. 
\begin{figure}[ht!]
\centering
%[width=90mm]
\includegraphics{scenario_normal.png}
\caption{Normal scenario}
\label{overflow}
\end{figure}

\subsection{Abnormal scenario}
The abnormal scenario is some kind of scenario that differs from normal scenario by some unusual but acceptable behaviors. In our case abnormal scenario includes additional user behaviors. The first first abnormal behavior is that user Ivanov uses Wiki server. Is is abnormal, but not so much suspicious, because other users use it every day. The second abnormal behavior is that user Petrov started to use DB server, but in normal sceanrion nobody use it. This user  behavior is more suspicious and could be determine as an attempt to hack the system. The abnormal scenario is illustrated on the Figure 2: Abnormal scenario. The black solid line show normal user behaviors. The yellow solid line shows the first abnormal behavior connection to Wiki server. The red solid line show the second abnormal behavior that is suspicious could be determined as an attempt to hack the system. 
\begin{figure}[ht!]
\centering
%[width=90mm]
\includegraphics{scenario_abnormal.png}
\caption{Abnormal scenario}
\label{overflow}
\end{figure}

\subsection{Implementation}
The first part of the implementation is the deployment and configuration the test environments. We used FutureSOC server, because it has needed computer resources. First of all, the private network was created to isolate our Virtual Machines (VMs) from others VMs on FutureSOC server with WMware ESXi. Then all machines was deployed and configured as described in the section "Environment description". We used Windows Server 2012 for Domain controller, Windows Server 2003 for Wiki server and DB server, Windows 7 Pro for user computers including admin's computer. 

The simulation of user behaviors is implemented by programming language Python with additional libraries. We call the python script for simulation of user behaviors as a simulator. To connect from simulator to VMs the virtual Network Computing (VNC) protocol was chose, because ESXi supports the VNC protocol. But it is needed to enable the VNC features on ESXi server and specified port for a vnc connection for each VM. To connect to VM by VNC we used vncdotool library ([!? link  https://github.com/sibson/vncdotool]). The library supports not only connecting to VMs by VNC, but also it can make screeshots of the VM desktop. The feature of making sreenshot of the VM desktop is used for checking a state of the VM.  The simulator describe the algorithm of user behavior for connection to VM, login into OS, run RDP connection to Wiki or DB server. We predefined some common state to determine the current state of VM during the simulation by comprising histograms of images. By comparison images and determine the current state of VM we can specify activities that are needed to perform. For example, if after comparison we determined that the state is "Logoff state" then we need pass CTL+ALT+DELETE command to VM to enable "Logon" state. In turn, in "Logon" state we can type username and password and press enter to enter to into the VM. The connection from the user computer to DB or Wiki server is implemented by running PowerShell Script included into VM. The script could take parameters and It means that we use one script for connection to both server by specifying certain ip address. The script establishes the RDP connection. 

In two above paragraph we described how we deploy, configure and connect to server. In this paragraph we describe how we describe the scenario of user behaviors. The description of scenarios is stored in csv files. The are 3 csv files. The first of them (computers.csv) describe the all computers which participate in the scenario. It contains the information about computer identifier, IP address, port specified for VNC connection, VNC password and type of operation system(OS).
The second files is called scenarios.csv. It describes the main user activities. The main user activity means the connection to the user computer. The file contains a computer ID referring to id in computer.csv file, username, password, session time, count of sessions and identifier of inner scenario. The third files is called inner-scenarios.csv. It describes the user activity after login into a user computer. For example, it could describe the connection to the wiki, the DB server or Domain controller or even connection to another user computer. There are two sets of csv files. The first set is used to perform normal scenario and the second is used to perform abnormal scenario 	
accordingly. Normal scenario takes about 3,5 hours and abnormal scenario takes about 5,5 hours.
 
%\end{document} 