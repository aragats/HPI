%%%%%%%%%%%%%%%%%%%%%%%%%%%%%%%%%%%%%%%%%
% Journal Article
% LaTeX Template
% Version 1.3 (9/9/13)
%
% This template has been downloaded from:
% http://www.LaTeXTemplates.com
%
% Original author:
% Frits Wenneker (http://www.howtotex.com)
%
% License:
% CC BY-NC-SA 3.0 (http://creativecommons.org/licenses/by-nc-sa/3.0/)
%
%%%%%%%%%%%%%%%%%%%%%%%%%%%%%%%%%%%%%%%%%

%----------------------------------------------------------------------------------------
%	PACKAGES AND OTHER DOCUMENT CONFIGURATIONS
%----------------------------------------------------------------------------------------

\documentclass[twoside]{article}

\usepackage{lipsum} % Package to generate dummy text throughout this template

\usepackage[sc]{mathpazo} % Use the Palatino font
\usepackage[T1]{fontenc} % Use 8-bit encoding that has 256 glyphs
\linespread{1.05} % Line spacing - Palatino needs more space between lines
\usepackage{microtype} % Slightly tweak font spacing for aesthetics

\usepackage[hmarginratio=1:1,top=32mm,columnsep=20pt]{geometry} % Document margins
\usepackage{multicol} % Used for the two-column layout of the document
\usepackage[hang, small,labelfont=bf,up,textfont=it,up]{caption} % Custom captions under/above floats in tables or figures
\usepackage{booktabs} % Horizontal rules in tables
\usepackage{float} % Required for tables and figures in the multi-column environment - they need to be placed in specific locations with the [H] (e.g. \begin{table}[H])
\usepackage{hyperref} % For hyperlinks in the PDF

\usepackage{lettrine} % The lettrine is the first enlarged letter at the beginning of the text
\usepackage{paralist} % Used for the compactitem environment which makes bullet points with less space between them

% ABSTRACT

\usepackage{abstract} % Allows abstract customization
%\renewcommand{\abstractnamefont}{\normalfont\bfseries} % Set the "Abstract" text to bold
%\renewcommand{\abstracttextfont}{\normalfont\small\itshape} % Set the abstract itself to small italic text
\renewcommand{\abstractname}{} % Empry abstract title
\renewenvironment{abstract}
 {\normalsize
  \begin{center}
  %\bfseries \abstractname\vspace{-.5em}\vspace{0pt}
  \end{center}
  \list{}{
    \setlength{\leftmargin}{.0cm}%
    \setlength{\rightmargin}{\leftmargin}%
  }%
  \item\relax}
 {\endlist}

% SECTIONS !!
\usepackage{titlesec} % Allows customization of titles
%\renewcommand\thesection{\Roman{section}} % Roman numerals for the sections
%\renewcommand\thesubsection{\Roman{subsection}} % Roman numerals for subsections
\renewcommand\thesection{\arabic{section}} % arabic numerals for the sections
\renewcommand\thesubsection{\arabic{section}.\arabic{subsection}} % arabic numerals for subsections

\titleformat{\section}[block]{\large\scshape}{\thesection.}{1em}{} % Change the look of the section titles
%\titleformat{\section}[block]{\large\scshape\centering}{\thesection.}{1em}{} % Change the look of the section titles
%raggedright
\titleformat{\subsection}[block]{\large}{\thesubsection.}{1em}{} % Change the look of the section titles

%VARIABLES

\newcommand{\myAuthorName}{Aragats Amirkhanyan}
\newcommand{\myUni}{Hasso Plattner Institute}
\newcommand{\myAuthorEmail}{aragats.amirkhanyan@hpi.de}
\newcommand{\myArticleTitle}{Security analysis based on the test environment}

\usepackage{fancyhdr} % Headers and footers
\pagestyle{fancy} % All pages have headers and footers
\fancyhead{} % Blank out the default header
\fancyfoot{} % Blank out the default footer
\fancyhead[C]{\myArticleTitle}
%$\bullet$ November 2012 $\bullet$ Vol. XXI, No. 1 % Custom header text
\fancyfoot[RO,RE]{Fall 2014 Workshop\thepage} % Custom footer text

%Header and footer Lines
\renewcommand{\headrulewidth}{0.4pt}% Default \headrulewidth is 0.4pt
\renewcommand{\footrulewidth}{0.4pt}% Default \footrulewidth is 0pt

%----------------------------------------------------------------------------------------
%	TITLE SECTION
%----------------------------------------------------------------------------------------

\title{\vspace{-15mm}\fontsize{24pt}{10pt}\selectfont\textbf{\myArticleTitle}} % Article title

\author{
\large
\textsc{\myAuthorName}\\
%\thanks{A thank you or further information}\\[2mm] % Your name
\normalsize \myUni \\ % Your institution
\normalsize \href{mailto:\myAuthorEmail}{\myAuthorEmail} % Your email address
\vspace{-10mm}
}
\date{}

%----------------------------------------------------------------------------------------

\begin{document}
\large %Font size

\maketitle % Insert title

\thispagestyle{fancy} % All pages have headers and footers

%----------------------------------------------------------------------------------------
%	ABSTRACT
%----------------------------------------------------------------------------------------

\begin{abstract}

\noindent 
\large %Font size
Research in the field of Information Security is always faced with the problem of lack of data or a suitable environment. This problem is not limited to research in the field of information security, but also for other areas. For research in the field of information security - is a big problem. Researchers have to find solutions for generate data, constructing a suitable environment, etc. A lot of affords to start the research. In the report, we consider examples of how we in the security group of the chair "Internet Technologies and Systems" solve similar problems. We consider the typical ways of solving them and consider a prototype project that we could help to simplify the implementation of research.
%\lipsum[1] % Dummy abstract text

\end{abstract}

%----------------------------------------------------------------------------------------
%	ARTICLE CONTENTS
%----------------------------------------------------------------------------------------

%\begin{multicols}{2} % Two-column layout throughout the main article text

\section{Introduction}

%\lettrine[nindent=0em,lines=3]{L} orem ipsum dolor sit amet, consectetur adipiscing elit.
%\lipsum[2-3] % Dummy text
Many research projects come from the industrial companies. It is seems that if the company is interested in success result of the research, the should provide all needed information, data, their environment. But they do not it. And obviously, the reason has a secure aspect. Most of researchers have to spend a big part of their time for preparing or generate data, deploy research environments and so on. The lack of the data is a big problem academic world. The preparation stage has quite big overhead of the research work, which could be overcome in some cases. It is not a secret that many researchers do the almost same researches which require the same data, the same environment. And every time they have to invent new bicycle. 

In this report we mostly speak about the problems of the security researches. We mention certain problems which we impact during the security research and the way of solving the problems. The report include the overview the prototype application which the main goal is to simplify the preparation stage of the research and make possible to overcome the research overhead. Despite the fact that the report contains information about security research all of that could be used for solving problems in the research project of other areas.      


\section{Deploy test environment}
Many security network and software analysis research start with preparing the test environment. The test environment is deployed in the most cases as the virtual network with configured hosts, routers, networks and all others resources including users account. The test environment could be deployed on the local computer by using the software for virtualization like the VirtualBox, VmWare Workstation and others or could be deployed on the remote server with installed hypervisor software like VMware vSphere Hypervisor.          

The common way to create the developing environment is to do it manually. You have to download image of operation system, install it on the Virtual Machine, configure it, install necessary softwares. The developing environment is usually more complex just one virtual machine, so you have to do the same step several. In some cases you can just clone virtual machine, but you still have to do a lot of manual work.       
  
There are some software which can simplify the process of creating the developing/test environment. One of the them is Vagrant. Vagrant allows to create and configure lightweight, reproducible, and portable development environments. The configured environment could be reused. Vagrant project has the relative project which is called VagrantCloud. The project is hosted on the https://vagrantcloud.com. It is some kind of the catalog of prepared environments. The prepared developing environment is called box.  People could share with community their boxes. Everyone could find the appropriate environment and reuse it instead of configuring new one. Vagrant allows to deploy the environment into the local VirtualBox, VMware Workstation, Aamazon Web Service (AWS). It means researchers are quite free in using the platform. The features of the Vagrant are not bound only by running the environment. The application a lot of give capabilities including synchronizing between the host and guest machine, integration with Chef[*], Puppet[*] and other. Learn more on the official website http://www.vagrantup.com/ 

The process of creating the development/test environment can by simplify by cloud providers. Many cloud providers provide capability easy to install and run virtual machine with any operation systems, configure the network and many other features. The most popular is Amazon Web Service (AWS). AWS is Infrastructure-as-a-Service. AWS provides huge amount of service which could solve any problem.  You could combine any infrastructure service to create the certain development environment. I mentioned just AWS, but the market of cloud providers grows extremely so it is possible find any other. As Vagrant AWS also provide capability to save configuration of the development environment, distribute it and reuse.

But we are talking about the researches with security aspects. It means that results and the process of the research in the most cases must secret. In this case it is not possible to use public cloud providers. Here opensource communities come to help us. There si not sense to tell that opensource communities grow extremely. A lot of big IT companies invest into the opensource projects. Many of these opensource projects are widely spread. They are use everywhere. So it does not go past the cloud technologies. There several opensource cloud platform which could be used for creating own IaaS, PaaS and other. Some of them: OpenStack, OpenNebula,  OpenShift Origin and so on. So clouds become private. For us it means that we could create our test/development environments by using flexibility and capability of cloud services. But still there are some problems and overhead of creating the test/development environments. It could be solved by using the PaaS, but PaaS's are usually designed for specific task mostly for running the infrastructure for web applications in Java, PHP, Ruby and so on. No one does provide flexible platform as a service based on all functionality of infrastructure as a service. We would like to call it Platform as a Infrastructure (PaaI). Jeslastic[*] uses this definition for describing its service, but they include other meaning into this definition.     

        


\section{Simulation of user behaviors}
dds  
%------------------------------------------------

\section{Methods}

Maecenas sed ultricies felis. Sed imperdiet dictum arcu a egestas. 
\begin{compactitem}
\item Donec dolor arcu, rutrum id molestie in, viverra sed diam
\item Curabitur feugiat
\item turpis sed auctor facilisis
\item arcu eros accumsan lorem, at posuere mi diam sit amet tortor
\item Fusce fermentum, mi sit amet euismod rutrum
\item sem lorem molestie diam, iaculis aliquet sapien tortor non nisi
\item Pellentesque bibendum pretium aliquet
\end{compactitem}
\lipsum[4] % Dummy text

%------------------------------------------------

\section{Results}

\begin{table}[H]
\caption{Example table}
\centering
\begin{tabular}{llr}
\toprule
\multicolumn{2}{c}{Name} \\
\cmidrule(r){1-2}
First name & Last Name & Grade \\
\midrule
John & Doe & $7.5$ \\
Richard & Miles & $2$ \\
\bottomrule
\end{tabular}
\end{table}

\lipsum[5] % Dummy text

\begin{equation}
\label{eq:emc}
e = mc^2
\end{equation}

\lipsum[6] % Dummy text

%------------------------------------------------

\section{Discussion}

\subsection{Subsection One}

\lipsum[7] % Dummy text

\subsection{Subsection Two}

\lipsum[8] % Dummy text

%----------------------------------------------------------------------------------------
%	REFERENCE LIST
%----------------------------------------------------------------------------------------

%\begin{thebibliography}{99} % Bibliography - this is intentionally simple in this template
%
%\bibitem[1]{Figueredo:2009dg}
%Figueredo, A.~J. and Wolf, P. S.~A. (2009).
%\newblock Assortative pairing and life history strategy - a cross-cultural
%  study.
%\newblock {\em Human Nature}, 20:317--330.
% 
%\end{thebibliography}

\input{mylib.bbl}
\bibliographystyle{IEEEtran}
%\bibliographystyle{ieeetr}
\bibliography{mylib}


%----------------------------------------------------------------------------------------

% Multicolumn END
%\end{multicols}

%\bibliography{mylib}

\end{document}
