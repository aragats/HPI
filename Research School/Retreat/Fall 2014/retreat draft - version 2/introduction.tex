\section{Introduction}

The scientific world has some problems that slow down the research progress. Some of them are obvious. One of them is the lack of data. Researchers need data for research. The huge amount of data belongs to industrial companies, but they do not hurry to share them because of security and privacy reasons. Researchers have to generate and simulate data. Further, many researchers need to configure a test network environment for research. It means that they have to run virtual machines, install software applications and configure the network. The third problem is that researchers need to simulate some activities on the test environment such as user behavior, open connections, running services and so on. These problems are common to all researchers, but for researchers in the security analytic and network security areas, these issues become challenges. The data for security research is more inaccessible than others. For example, domain controller logs often contain personal information, e.g. user Id, time of login and logout events. This information is often an object of data privacy and should be anonymized before exporting for analysis. And it actually means that almost there is not a chance to get this data. The test environments for research in the security area are usually more specific. Every often it is needed to have the potentially vulnerable environment, and that means that old and vulnerable software applications must be installed. This restriction complicates the ability to use many existing IT automation systems such as Chef and Puppet. The remainder  of the report covers ideas, ways of overcoming some mentioned issues and an overview of research projects.

%The remainder of the paper is structured as follow: 