\section{Simulation of user behaviors}
On past semester the Security Group of the chair "Internet Technologies and Systems" worked on the project "Machine Learning for Security Analytics powered by SAP HANA". Within this project the group aimed to implement and test the machine learning approach for security analytics based on SAP HANA. Under this approach the group focused on the analysis of user events, particularly login and logout events. For testing of machine learning analytics module, we have generated several simple brute-force attacks. Also there was the idea to try the machine learning approach powered on SAP HANA for security analytics with more sophisticated cases. But the main problem of trying the approach with sophisticated cases is the lack of data. The lack of data is common problem of the academic world. Mostly, only industrial companies have such data in needed amount but they do not share them because of some obvious reasons such as security and privacy. For example, login and logout events contain information about users and this information can not be passed to third parties. We are faced with the task to implement a tool to simulate particular user scenarios, user behaviors and as result generate the necessary data for further analysis.
The scenario in our case is the description of the network infrastructure, information about users and a description of user behaviors. To analysis user behaviors by the machine learning approach we must have a normal scenario and an abnormal scenario.  
  
\subsection{Environment description}
We start with designing and describing the network. The main node of the network is the domain controller (DC). On the domain controller we installed Widows Sever 2012. Additionally, we added two virtual machines (VMs) with Windows Server 2013 that are used as the wiki server and the database server accordingly. Also the network contains 4 client virtual machines with Windows 7 Professional 64-bit. To complete the installation we created four user accounts and one of them is the admin user who has access to any computer. All the above information is presented below in the form of compact lists

Description of the network:
\begin{compactitem}
\item 4 client computers with Windows 7 Professional 64-bit.
\item Domain controller with Widows Server 2012.
\item Wiki server with Windows Server 2003.
\item Database server with Windows Server 2003.
\end{compactitem}
Users: 
\begin{compactitem}
\item Ivanov
\item Petrov
\item Smirnov
\item Admin
\end{compactitem}
     
\subsection{Normal scenario}
All users has access to client computers, the wiki server and the database server. Admin has access to any computer. In the case of the normal scenario users log into client computers by their credentials. They do it several times per day. Petrov and Smirnov during the workday visit the wiki server. Ivanov does not visit the wiki server despite that he has access. All users have access to the database server, but no one uses it. Admin usually logs into to his computer and during the workday he logs into to the domain controller, the database server and the wiki server by the remote desktop connection (RDP). Other users cab not log into the domain controller by the RDP because they do not have access. The description of the normal behavior you can see on the Figure 1. All black lines show the access to the computers. The black solid lines illustrate the access which are used by the users. The black dash lines show access which are not used by users in the normal behavior despite that they have access. 
\begin{figure}[ht!]
\centering
%[width=90mm]
\includegraphics[width=\textwidth]{scenario_normal.png}
\caption{Normal scenario}
\label{overflow}
\end{figure}

\subsection{Abnormal scenario}
The abnormal scenario is the scenario that differs from the normal scenario by some unusual but acceptable behavior. In our case the abnormal scenario includes additional users behavior. The first abnormal behavior is that the user Ivanov uses the wiki server. It is abnormal but not suspicious because other users use it every day. The second abnormal behavior is that the user Petrov uses the database server but in the normal scenario no one uses it. This user behavior is more suspicious and could be determined as an attempt to hack the system. The abnormal scenario is illustrated on the Figure 2. The black solid lines show normal users behavior. The yellow solid line shows the first abnormal behavior. The red solid line shows the second abnormal behavior that is suspicious and could be determined as an attempt to hack the system. 
\begin{figure}[ht!]
\centering
%[width=90mm]
\includegraphics[width=\textwidth]{scenario_abnormal.png}
\caption{Abnormal scenario}
\label{overflow}
\end{figure}

\subsection{Implementation}
The first part of the implementation is the deployment and the configuration of the test environment. To set up the network we used FutureSOC server with installed VMware ESXi, because it has necessary resources. Firstly, we created the network to isolate our virtual machines from others hosted on the same ESXi server. Afterwards all machines were deployed and configured as described in the section "2.1 Environment description". We used Windows Server 2012 for the domain controller, Windows Server 2003 for the wiki and the database server and Windows 7 Professional 64-bit for client computers. 

The simulation of user behavior is implemented on the python programming language with additional libraries. The program is called the simulator. The simulator uses the virtual network computing protocol (VNC) to connect to the virtual machines as the ESXi server supports the VNC protocol. But it is needed to enable VNC on the ESXi server and specify the VNC port for each virtual machine. To connect to virtual machines using the VNC protocol, we used the vncdotool library\footnote{GitHub repository: https://github.com/sibson/vncdotool}. The library also can take a screenshot of the remote desktop. We use the screenshot for checking a state of the virtual machine. The simulator describes an algorithm of user behavior such as the connection to the virtual machine, the entrance to the operation system, run the remote desktop protocol (RDP) connection to the wiki or the database server. We predefined some common states to determine the current state of the virtual machine during the simulation by comparing the image histograms. Once we determined the current state of the virtual machine, we can specify the activities to be undertaken in each case according to the scenario description such as send the ctrl+alt+delete command to the virtual machine, enter the username and the password, open the RDP connection and others. The RDP connection from the client computer to the database or the wiki server is implemented by running the PowerShell script hosted on the virtual machine. The script can accept parameters and it means that we can use the script to establish the connection with both servers by specifying the IP address.

In this paragraph, we show how we describe the scenario of user behavior. The scenario descriptions are stored in csv files. There are three csv files. The first of them is called computers.csv. It describes all computers that are involved in the scenarios. It contains the information about the computer identifier, the IP address or the host, the VNC port, the VNC password and the type of operation system.
The second file is called scenarios.csv. It describes main user activities. The main user activity means the connection to the client computer. The file contains the computer identifier referring to the identifier in the computer.csv file, the username, the password, the session time, the count of sessions and the identifier to the inner scenario. The third files is called inner-scenarios.csv. It describes the actions of users after logging into the client computer. For example, it could describe the connection to the wiki server, database server or domain controller or even the connection to another client computer. The file contains the identifier, the host, the username, the password and the session time. There are two sets of csv files. The first set is used to perform the normal scenario and the second is used to perform the abnormal scenario, respectively. The described relations between csv files are illustrated on the Figure 3.
 	
\begin{figure}[ht!]
\centering
%[width=90mm]
\includegraphics[width=\textwidth]{scenario_relations.png}
\caption{Scenario relations}
\label{overflow}
\end{figure}


The normal scenario takes about 3,5 hours with 41 login and logout events into client computers and 25 RDP connections to the wiki and database servers. The abnormal scenario takes about 5,5 hours with 45 login and logout events to client computers and 34 RDP connections to the the wiki and database servers.
 
%\end{document} 