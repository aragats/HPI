\begin{abstract}


The report summarizes my activities of the past semester at the Security Group of the chair "Internet Technologies and Systems". The report contains my views about research works and reflects certain activities in research projects.
I started the past semester with working in the project "Machine Learning for Security Analytics powered by SAP HANA". Under this project I aimed to write the application to simulate users behavior. The report reflects an overview of the project, a description of the test network environment which was used for researching, a description of scenarios of the simulation and implementation details.
The next step was learning the research project called Security Lab Generator.
Security Lab Generator (SLG) is the research project that is aimed to configure, deploy, test, monitor, analyze the network environments regarding security issues. The report includes an overview of the project, an overview of SLG components and use cases of the project. 
Based on the experience gained from working and learning in the previous projects I share my view on major issues of the academic world. The definition Open Research Cloud is introduced. The report shows some related works with the Open Research Cloud concept. A part of the report is devoted to learn an open-source Infrastructure as a Service called OpenStack and deploy some service on a test HPI server. The reports contains description of OpenStack architecture deployed on the HPI server. 
In the end of the report I summarized and made proposals about researches in the security area on the cloud and I suggested to make first steps towards making researches on the cloud.

\end{abstract}