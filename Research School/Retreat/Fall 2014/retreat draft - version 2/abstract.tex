\begin{abstract}


The report summarizes my activities of the past semester at the Security Group of the chair "Internet Technologies and Systems". The report contains my views about research works and reflects certain activities in research projects.
I started the past semester with working in the project "Machine Learning for Security Analytics powered by SAP HANA". Within this project the security group aimed to implement and test the machine learning approach for security analytics based on SAP HANA. Under this project I aimed to write the application to simulate users behavior and as a result generate more sophisticated data for the machine learning approach. The report reflects the overview of the project, the description of the test network environment that was used for the research, the description of scenarios of the simulation and implementation details.

The next step was learning the research project called Security Lab Generator.
Security Lab Generator (SLG) is the research project that is aimed to configure, deploy, test, monitor, analyze the network environments regarding security issues. The report includes the overview of the project, the overview of SLG components and use cases of the project. 

The rest part of the report reflects learning OpenStack project. OpenStack is an open-source infrastructure as a service. I was learning OpenStack and I deployed some services on a test HPI server. The reports contains the description of the OpenStack architecture deployed on the HPI server. As the result the definition Open Research Cloud is introduced. The report shows some related works with the Open Research Cloud concept.  

In the end of the report I summarized and made proposals about research in the security area on the cloud. I suggested to make first steps towards research on the cloud. As an example I suggested to migrate SLG on the cloud technologies. 


\end{abstract}